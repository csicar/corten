%% Beispiel-Präsentation mit LaTeX Beamer im KIT-Design
%% entsprechend den Gestaltungsrichtlinien vom 1. August 2020
%%
%% Siehe https://sdqweb.ipd.kit.edu/wiki/Dokumentvorlagen

%% Beispiel-Präsentation
\documentclass{sdqbeamer} 
\usepackage[cache=false]{minted}
\usemintedstyle{solarized-light}
% \newminted{rust}{escapeinside=~~, autogobble, style=solarized-dark, fontsize=\footnotesize, linenos}
\newminted{rust}{autogobble, fontsize=\small, escapeinside=@@}


\usepackage{varwidth}
\usepackage{tikz}
\usetikzlibrary{calc, patterns, tikzmark, arrows, decorations, shapes.callouts, decorations.pathreplacing, decorations.pathmorphing, external}
% \tikzexternalize
% \setbeameroption{show notes on second screen}
\usepackage[overridenote, notesposition=right]{pdfpc}

%% Titelbild
\titleimage{banner_2020_kit}

%% Gruppenlogo
\grouplogo{}

%% Gruppenname
\groupname{Application-oriented Formal Verification}

% Beginn der Präsentation

\title[LiquidRust]{LiquidRust}
\subtitle{Embracing Mutability on Refinement Types using Rust's Ownership Model} 
\author[Carsten Csiky]{Carsten Csiky}

\date[24.\,2.\,2022]{24. Februar 2022}

% Literatur 
 
% \usepackage[citestyle=authoryear,bibstyle=numeric,hyperref,backend=biber,datatype=bibtex]{biblatex}
\usepackage[citestyle=authoryear,bibstyle=numeric,hyperref,backend=biber]{biblatex}
\addbibresource{lib.bib}
\bibhang1em

% https://tex.stackexchange.com/a/445690
\makeatletter
\pgfset{
  /pgf/decoration/randomness/.initial=2,
  /pgf/decoration/wavelength/.initial=100
}
\pgfdeclaredecoration{sketch}{init}{
  \state{init}[width=0pt,next state=draw,persistent precomputation={
    \pgfmathsetmacro\pgf@lib@dec@sketch@t0
  }]{}
  \state{draw}[width=\pgfdecorationsegmentlength,
  auto corner on length=\pgfdecorationsegmentlength,
  persistent precomputation={
    \pgfmathsetmacro\pgf@lib@dec@sketch@t{mod(\pgf@lib@dec@sketch@t+pow(\pgfkeysvalueof{/pgf/decoration/randomness},rand),\pgfkeysvalueof{/pgf/decoration/wavelength})}
  }]{
    \pgfmathparse{sin(2*\pgf@lib@dec@sketch@t*pi/\pgfkeysvalueof{/pgf/decoration/wavelength} r)}
    \pgfpathlineto{\pgfqpoint{\pgfdecorationsegmentlength}{\pgfmathresult\pgfdecorationsegmentamplitude}}
  }
  \state{final}{}
}
\tikzset{xkcd/.style={decorate,decoration={sketch,segment length=0.5pt,amplitude=0.5pt}}}
\tikzset{xkcd-little/.style={decorate,decoration={sketch,segment length=0.5pt,amplitude=0.25pt}}}
\makeatother

\pgfmathsetseed{1}


\definecolor{c453A62}{RGB}{69,58,98}
\definecolor{c5E5086}{RGB}{94,80,134}
\definecolor{c8F4E8B}{RGB}{143,78,139}

\begin{document}

%Titelseite
\KITtitleframe

%Inhaltsverzeichnis
% \begin{frame}{Inhaltsverzeichnis}
% \tableofcontents
% \end{frame}

\section{Motivation}

\begin{frame}{Motivation}
  \centering
  \begin{tikzpicture}[xscale=.2, yscale=.2]
    \coordinate (good) at (90:8.5);
    \coordinate (bad) at (-90:8.5);
    \clip (-40, -15) rectangle (40, 15);
    \begin{scope}[very thick, 
        every path/.style={xkcd}
        ]
      \draw circle [radius=10, rotate=90];
      % \filldraw[fill=green!20,draw=green!50!black] (0, 0) -- (80:11) arc [start angle=0, end angle=100, radius=11] -- cycle;
      \filldraw[rotate=90, fill=green!20,draw=green!50!black] (0,0) -- (-15:10.5)
          arc [start angle=-15, end angle=15, radius=10.5] -- cycle;
      % \draw (0, 0) -- (100:12);
    \end{scope}

    \node[anchor=west, align=left] (good_label) at (16,5) {Actual Goal};
    \node[anchor=east, align=right] (bad_label) at (-13,-5) {$h(\texttt{var}) = h'(\texttt{var})$\\ Variable unchanged between heaps};
    
    \draw[<-, ultra thick, cyan!60!black, dashed] (good) to[out=45, in=150] (good_label);
    \draw[<-, ultra thick, cyan!60!black, dashed] (bad) to[out=-135, in=-5] (bad_label);
  \end{tikzpicture}
\end{frame}

\section{Foundations}

\begin{frame}
  \centering

  \begin{tikzpicture}[x=1pt, y=1pt, remember picture]
    \clip (-235 , -40) rectangle (235 , 190);

    \node[minimum size=40] (rust_character) at (160, 0) {};
    \node[minimum size=40] (haskell_character) at (-160, 0) {};

    % \draw[help lines, black!5, very thin, step=10] (-300,-50) grid (300, 170);

    
    \node[] (rust_img) at ($(rust_character) + (0, 20)$) {};
    \node[] (haskell_img) at ($(haskell_character) + (0, 20)$) {};

    \scoped[shift=(haskell_img)]
      \begin{scope}[yscale=-1, scale=0.23, xshift=-70, yshift=10] 
        \path[fill=c453A62] (0.0000,180.5910) -- (60.2350,90.2970) -- (0.0000,0.0000) --
  (45.1760,0.0000) -- (105.4120,90.2970) -- (45.1760,180.5910) -- cycle;
\path[fill=c5E5086] (60.2350,180.5910) -- (120.4710,90.2970) -- (60.2350,0.0000)
  -- (105.4120,0.0000) -- (225.8770,180.5910) -- (180.7060,180.5910) --
  (143.0610,124.1590) -- (105.4090,180.5910) -- cycle;
\path[fill=c8F4E8B] (205.8040,127.9200) -- (185.7250,97.8200) --
  (256.0000,97.8200) -- (256.0000,127.9220) -- (205.8040,127.9220) --
  (205.8040,127.9200) -- cycle(175.6860,82.7750) -- (155.6080,52.6750) --
  (256.0000,52.6750) -- (256.0000,82.7750) -- (175.6860,82.7750) -- cycle;

      \end{scope};

    \onslide<4- > {
    \scoped[shift=(rust_img)]
      \begin{scope}[yscale=-1, scale=0.1, xshift=-240] 
        \path[fill] (508.5200,249.7500) -- (486.7000,236.2400) .. controls
  (486.5300,234.2400) and (486.3600,232.3100) .. (486.1500,230.3600) --
  (504.8700,212.8600)arc(46.914:-69.444:7.350) -- (478.4300,191.6100) ..
  controls (477.8900,189.7300) and (477.3500,187.8300) .. (476.7600,185.9700) --
  (491.7600,165.1400)arc(35.667:-80.751:7.350) -- (461.5500,149.4500) ..
  controls (460.6500,147.7200) and (459.7600,146.0000) .. (458.8200,144.3000) --
  (469.5000,120.8800)arc(24.471:-92.029:7.350) -- (436.7300,111.4000) ..
  controls (435.5367,109.9200) and (434.3333,108.4533) .. (433.1200,107.0000) --
  (439.0000,81.8400)arc(13.135:-103.135:7.360) -- (405.0000,78.9300) .. controls
  (403.5533,77.7100) and (402.0867,76.5067) .. (400.6000,75.3200) --
  (401.5100,49.5000)arc(2.426:-114.488:7.350) -- (367.7000,53.2300) .. controls
  (366.0000,52.2900) and (364.2700,51.3900) .. (362.5500,50.5000) --
  (358.4000,25.0800)arc(350.751:234.333:7.350) -- (326.0000,35.2600) .. controls
  (324.1400,34.6700) and (322.2500,34.1300) .. (320.3600,33.5900) --
  (311.3600,9.5900)arc(339.444:223.086:7.350) -- (281.6100,25.8700) .. controls
  (279.6600,25.6600) and (277.7000,25.4900) .. (275.7300,25.3200) --
  (262.2500,3.4800)arc(328.249:211.751:7.350) -- (236.2400,25.3000) .. controls
  (234.2400,25.4700) and (232.3100,25.6400) .. (230.3600,25.8500) --
  (212.8600,7.1300)arc(316.914:200.556:7.350) -- (191.6100,33.5700) .. controls
  (189.7200,34.1200) and (187.8200,34.6500) .. (185.9500,35.2500) --
  (165.1300,20.2500)arc(305.667:189.249:7.350) -- (149.4400,50.4500) .. controls
  (147.7100,51.3500) and (145.9900,52.2400) .. (144.2800,53.1800) --
  (120.8800,42.5500)arc(294.488:177.574:7.350) -- (111.4100,75.3600) .. controls
  (109.9200,76.5500) and (108.4100,77.7500) .. (106.9900,78.9700) --
  (81.8400,73.0000)arc(283.135:166.865:7.360) -- (78.9300,107.0000) .. controls
  (77.7000,108.4500) and (76.5000,109.9300) .. (75.3100,111.4100) --
  (49.5000,110.5000)arc(271.726:214.069:7.420)arc(213.687:155.455:7.350) --
  (53.2200,144.3000) .. controls (52.2800,146.0000) and (51.3900,147.7300) ..
  (50.4900,149.4600) -- (25.0800,153.6000)arc(260.751:144.333:7.350) --
  (35.2900,185.9600) .. controls (34.7000,187.8300) and (34.1600,189.7300) ..
  (33.6100,191.6200) -- (9.6100,200.6200)arc(249.444:133.086:7.350) --
  (25.8900,230.3700) .. controls (25.6800,232.3200) and (25.5100,234.2800) ..
  (25.3400,236.2500) -- (3.4800,249.7500)arc(238.249:121.751:7.350) --
  (25.3000,275.7600) .. controls (25.4700,277.7600) and (25.6400,279.6800) ..
  (25.8500,281.6300) -- (7.1300,299.1300)arc(226.914:110.556:7.350) --
  (33.5700,320.3800) .. controls (34.1200,322.2700) and (34.6500,324.1600) ..
  (35.2500,326.0300) -- (20.2500,346.8600)arc(215.667:99.249:7.350) --
  (50.4600,362.5500) .. controls (51.3600,364.2700) and (52.2500,366.0000) ..
  (53.1900,367.6900) --
  (42.5600,391.1200)arc(204.545:146.313:7.350)arc(147.222:86.983:7.130) --
  (75.3300,400.6000) .. controls (76.5100,402.0800) and (77.7100,403.5467) ..
  (78.9300,405.0000) -- (73.0000,430.1600)arc(193.135:76.865:7.360) --
  (107.0000,433.0700) .. controls (108.4533,434.2900) and (109.9233,435.4933) ..
  (111.4100,436.6800) -- (110.4900,462.5000)arc(182.029:65.529:7.350) --
  (144.3100,458.7700) .. controls (146.0000,459.7100) and (147.7300,460.6000) ..
  (149.4500,461.5000) -- (153.6000,486.9200)arc(170.807:54.193:7.340) --
  (185.9700,476.7000) .. controls (187.8300,477.3000) and (189.7300,477.8300) ..
  (191.6200,478.3800) -- (200.6200,502.3800)arc(159.319:43.211:7.360) --
  (230.3700,486.1000) .. controls (232.3200,486.3100) and (234.2900,486.4800) ..
  (236.2500,486.6500) -- (249.7600,508.4700)arc(148.249:31.751:7.350) --
  (275.7700,486.6500) .. controls (277.7700,486.4800) and (279.7000,486.3100) ..
  (281.6500,486.0900) -- (299.1500,504.8200)arc(136.789:20.681:7.360) --
  (320.4000,478.3800) .. controls (322.2900,477.8300) and (324.1800,477.3000) ..
  (326.0500,476.7000) -- (346.8700,491.7000)arc(125.807:9.193:7.340) --
  (362.5600,461.5000) .. controls (364.2800,460.6000) and (366.0100,459.7100) ..
  (367.7100,458.7700) -- (391.1300,469.4500)arc(114.471:-2.029:7.350) --
  (400.6100,436.6800) .. controls (402.0900,435.4867) and (403.5567,434.2833) ..
  (405.0100,433.0700) -- (430.1600,439.0000)arc(103.135:-13.135:7.360) --
  (433.0700,405.0000) .. controls (434.2900,403.5533) and (435.4933,402.0867) ..
  (436.6800,400.6000) --
  (462.5000,401.5100)arc(92.559:33.237:7.230)arc(33.770:-24.469:7.350) --
  (458.7700,367.7000) .. controls (459.7100,366.0000) and (460.6000,364.2700) ..
  (461.5000,362.5500) -- (486.9200,358.4000)arc(80.751:-35.667:7.350) --
  (476.7100,326.0300) .. controls (477.3000,324.1600) and (477.8400,322.2700) ..
  (478.3800,320.3800) -- (502.3800,311.3800)arc(69.444:-46.914:7.350) --
  (486.1000,281.6300) .. controls (486.3100,279.6800) and (486.4800,277.7200) ..
  (486.6500,275.7600) -- (508.4700,262.2500)arc(58.249:-58.249:7.350) --
  cycle(357.5200,378.8300)arc(282.118:192.118:13.910) --
  (333.3600,425.1800)arc(65.629:114.913:187.510) --
  (169.3600,388.7800)arc(347.995:257.959:13.870) --
  (121.3900,384.8600)arc(135.904:143.605:187.380) -- (258.3000,365.6500) ..
  controls (260.0200,365.6500) and (261.1900,365.3600) .. (261.1900,363.7400) --
  (261.1900,309.5500) .. controls (261.1900,307.9800) and (260.0200,307.6400) ..
  (258.3000,307.6400) -- (213.4700,307.6400) -- (213.5200,273.2900) --
  (262.0000,273.2900) .. controls (266.4100,273.2900) and (285.6600,274.5700) ..
  (291.7900,299.1600) .. controls (293.7000,306.7100) and (297.9600,331.3000) ..
  (300.8500,339.1600) .. controls (303.7400,347.9800) and (315.4500,365.6200) ..
  (327.9500,365.6200) -- (407.0000,365.6200)arc(36.736:44.860:187.300) --
  cycle(383.2900,413.3200)arc(0.000:269.812:15.240) --
  (368.4400,398.0800)arc(-88.382:0.060:15.230) --
  cycle(157.6700,412.6400)arc(0.038:269.962:15.240) --
  (142.8700,397.3900)arc(-88.121:0.031:15.250) -- cycle(69.5700,234.1500) --
  (102.4000,219.5500)arc(66.104:-23.974:13.880) -- (102.6900,186.0000) --
  (129.2500,186.0000) -- (129.2500,305.7300) --
  (75.6500,305.7300)arc(164.110:186.180:187.650) --
  cycle(58.3100,198.0900)arc(179.962:269.962:15.240) --
  (74.0000,182.8400)arc(-88.383:179.977:15.240) -- cycle(213.4700,222.5800) --
  (213.5200,187.2600) -- (276.7800,187.2600) .. controls (280.0600,187.2600) and
  (299.8500,191.0300) .. (299.8500,205.8800) .. controls (299.8500,218.1700) and
  (284.6600,222.5800) .. (272.1700,222.5800) -- cycle(399.0000,306.7100) ..
  controls (389.2000,307.8400) and (378.3700,302.5900) .. (377.0000,296.6200) ..
  controls (371.2200,264.1300) and (361.6100,257.2200) .. (346.4300,245.2200) ..
  controls (365.2900,233.2700) and (384.8900,215.5800) .. (384.8900,191.9600) ..
  controls (384.8900,166.4400) and (367.4000,150.3700) .. (355.4900,142.4800) ..
  controls (338.7300,131.4800) and (320.2100,129.2500) .. (315.2200,129.2500) --
  (116.3200,129.2500)arc(221.829:259.298:187.490) --
  (244.6800,94.6600)arc(136.464:46.108:13.820) --
  (290.5400,70.1000)arc(280.610:330.310:187.510) --
  (400.9100,202.1000)arc(203.434:114.273:14.000) --
  (442.5900,235.7600)arc(-5.693:4.284:187.120) -- (423.7100,268.3000) ..
  controls (421.8000,268.3000) and (421.0200,269.5700) .. (421.0200,271.4300) --
  (421.0200,280.2500) .. controls (421.0000,301.0000) and (409.3100,305.5800) ..
  (399.0000,306.7100) -- cycle(240.0000,60.2100)arc(180.113:269.887:15.240) --
  (255.6600,45.0000)arc(-88.421:180.091:15.240) --
  cycle(436.8400,214.0000)arc(90.000:270.000:15.240) --
  (437.2800,183.5200)arc(-89.173:90.827:15.242) -- cycle;

      \end{scope};
    }

    \onslide<2- > {
    \node[draw, xkcd-little, rectangle callout, callout pointer shorten=20, callout absolute pointer={(haskell_character)}] (haskell_speech) 
        at ($(haskell_character) + (-15, 40)$) 
        {Mutability is the problem!};
    }
    \onslide<3> {
    \node[draw, xkcd-little, rectangle callout, callout pointer shorten=30, callout absolute pointer={(haskell_character)}] (refinement_types) 
      at ($(haskell_character) + (200, 70)$) 
      {\begin{varwidth}{\linewidth}
          \textbf{Refinement Types}
          \begin{itemize}
            \item Extension of the Type-System
            \item \texttt{i32} $\Rightarrow$ \texttt{\{ v: i32 | v > 0 \}}
          \end{itemize}
          \vspace{1em}
          \textbf{Type Checking}\\
          Given Type Context: \\
          $
            \Gamma = \{ f: \{ v: \tau_{param} \mid p(v) \} \to \{ u : \tau_{res} \mid r(u) \},
               a : \{ w: \tau_{arg} | q(w) \} \}
          $\\
          % Given $\Gamma = \{ f: \{ v: \tau_{param} \mid p(v) \} \to \{ u : \tau_{res} \mid r(u) \}, a : \{ w: \tau_{arg} | q(w) \} \}$\\
          Is \texttt{f(a)} type correct?
          \begin{itemize}
            \item[(I)] $\{ v : \tau_{arg} \mid p(v) \} \preceq \{ w : \tau_{param} \mid q(w) \}$
          \end{itemize}
          \vspace{1em}
          Subtyping judgment $\{ \tau | p \} \preceq \{ \tau' | q \}$
          \begin{itemize}
            \item[(I)] Base types $\tau, \tau'$ unify
            \item[(II)] $\forall v .\ \tau. p(v) \Rightarrow q(v)$
          \end{itemize}
      \end{varwidth}};
    }

    \onslide<4- > {
    \node[draw, xkcd-little, rectangle callout, callout pointer shorten=20, callout absolute pointer={(rust_character)}] (rust_speech) 
        at ($(rust_character) + (15, 40)$) 
        {\begin{varwidth}{\linewidth}
          No quite: Mutable \\\textit{aliasing} is the problem!
        \end{varwidth}};
    }

    \onslide<5> {
    \node[draw, xkcd-little, rectangle callout, callout pointer shorten=30, callout absolute pointer={(rust_character)}]
      (ownership_model) 
      at ($(rust_character) + (-150, 70)$) 
      {\begin{varwidth}{\linewidth}
          Every lexical scope:
          \begin{itemize}
            \item tracks "Permissions" 
            \item for every lexically visible variable
          \end{itemize}
          Possible Variable Permissions:
          \begin{enumerate}
            \item Owner \texttt{v}
              \begin{itemize}
                \item read
                \item write
                \item can transfer ownership, if no outstanding\\ borrows at time of transfer
              \end{itemize}
            \item immutable Reference \texttt{\&v}
              \begin{itemize}
                \item read
                \item guarantee: no writes
                \item possibly other readers
              \end{itemize}
            \item mutable Reference \texttt{\&mut v}
              \begin{itemize}
                \item read and write
                \item no other reader or writer
              \end{itemize}
          \end{enumerate}
      \end{varwidth}};
    }
  \end{tikzpicture}
\end{frame}

\section{Master Thesis}

\begin{frame}{Rust + Refinement Types?}
  How can Refinement Types be adapted for mutable Languages leveraging Rust's  Ownership Model?
  \vspace{1em}\\
  \textbf{Goal}
  \begin{itemize}
    \item design a decidable type system with refinement types for functional verification in Rust.
    \item implement a proof of concept for automatic verification / type checking.
    \item backward-compatible to Rust
    \item limit to safe, (non-higher-order?) Rust
    \item no liquid type inference
  \end{itemize}
\end{frame}

\begin{frame}[fragile=singleslide]{How?}
  \begin{columns}[T] % align columns
    \begin{column}{.6\textwidth}
      \begin{itemize}
        % \item define syntax of refinement language
        \item adapt refinement types semantics to rust, especially for mutability in Rust's Ownership Model
        \item prototype translation to verification backend (e.g. z3, prusti)
        \item evaluate the verification and the expressiveness of the refinement language on minimal examples
      \end{itemize}
    \end{column}%
    \hfill%
      \begin{column}{.38\textwidth}
        \begin{rustcode}
          fn push_all(
              a: &mut Vec<i32>, 
              b: &Vec<i32>) {
                ..
          }

          fn client() -> i32 {
            let mut a = vec![1, 2];
            let b = vec![2, 3];
            push_all(a, b);
            return a[5]; // type error!
          }
        \end{rustcode}
      \end{column}%
    \end{columns}
  
  
  % TODO! Bsp concat swap
\end{frame}

\section{Related Work}

\begin{frame}{Related Work}
  \begin{itemize}
    \item Property Types in Java\cite{lanzinger_property_2021}
      \begin{itemize}
        \item Adaptation of Refinement Types for Java
        \item Limited to immutable (\texttt{final}) subset
      \end{itemize}
    \item Prusti\cite{astrauskas_leveraging_2019}
      \begin{itemize} 
        \item Heavy-Weight functional verification for rust with semantics for \texttt{unsafe}.
        \item Seperation Logic
      \end{itemize}
    \item Async Liquid Types\cite{kloos_asynchronous_2015}
      \begin{itemize} 
        \item for Refinement Types + Mutability in OCaml
        \item cannot take advantage of Rust's Ownership model
      \end{itemize}
    \item RustHorn\cite{matsushita_rusthorn_2020}
      \begin{itemize} 
        \item constrained horn clauses based verification
        \item \texttt{\&mut} handling interesting and should be adapted for LiquidRust
      \end{itemize}
  \end{itemize}
\end{frame}


\appendix
\beginbackup



\begin{frame}{Literatur}

  \printbibliography
\end{frame}



\begin{frame}[fragile=singleslide]{Motivation}
  \note{
      - **Beispiel**: Nachbau eines Teiles d. Campus Management Systems
    \\
      - Aufgabe: Nach Prüfung neue note eintragen Student über Status informieren
    }
  % \begin{rustcode}
  %   /// Add `b` to end of Vec `as`
  %   fn push(as: &mut Vec<i32>, b: i32) {..}
  %   /// Get highest value. Returns `None` for empty Vec
  %   fn maximum(as: &Vec<i32>) -> Option<i32> {..}

  %   fn update_exam_tries(tries: &mut Vec<i32>, new_try: i32) {
  %     push(&mut tries, new_try);

  %     match maximum(ties) {
  %       None => panic!("impossible"), // `tries` must at least contain `new_try`
  %       Some(max_mark) if max_mark < 4    => println!("puh"),
  %       _              if tries.len() > 3 => println!("Härtefallantrag?"),
  %       _                                 => println!("versuch's nochmal"),
  %     }
  %   }
  % \end{rustcode}
\end{frame}

\subsection{Erster }
\begin{frame}{Blöcke}{in den KIT-Farben}
  \begin{columns}
    \column{.3\textwidth}
    \begin{greenblock}{Greenblock}
      Standard (\texttt{block})
        \end{greenblock}
    \column{.3\textwidth}
    \begin{blueblock}{Blueblock}
      = \texttt{exampleblock}
        \end{blueblock}
    \column{.3\textwidth}
    \begin{redblock}{Redblock}
      = \texttt{alertblock}
        \end{redblock}
  \end{columns}
  \begin{columns}
    \column{.3\textwidth}
        \begin{brownblock}{Brownblock}
        \end{brownblock}
    \column{.3\textwidth}
        \begin{purpleblock}{Purpleblock}
        \end{purpleblock}
    \column{.3\textwidth}
        \begin{cyanblock}{Cyanblock}
        \end{cyanblock}
  \end{columns}
  \begin{columns}
    \column{.3\textwidth}
        \begin{yellowblock}{Yellowblock}
        \end{yellowblock}
    \column{.3\textwidth}
        \begin{lightgreenblock}{Lightgreenblock}
        \end{lightgreenblock}
    \column{.3\textwidth}
        \begin{orangeblock}{Orangeblock}
        \end{orangeblock}
  \end{columns}
  \begin{columns}
    \column{.3\textwidth}
        \begin{grayblock}{Grayblock}
        \end{grayblock}
    \column{.3\textwidth}
    \begin{contentblock}{Contentblock}
      (farblos)
    \end{contentblock}
    \column{.3\textwidth}
  \end{columns}
\end{frame}
    
\subsection{Zweiter Unterabschnitt}
\begin{frame}{Auflistungen}
  Text
  \begin{itemize}
    \item Auflistung\\ Umbruch
    \item Auflistung
    \begin{itemize}
      \item Auflistung
      \item Auflistung
    \end{itemize}
  \end{itemize}
\end{frame}

\section{Zweiter Abschnitt}

\begin{frame}
        Bei Frames ohne Titel wird die Kopfzeile nicht angezeigt, und  
    der freie Platz kann für Inhalte genutzt werden.
\end{frame}

\begin{frame}[plain]
    Bei Frames mit Option \texttt{[plain]} werden weder Kopf- noch Fußzeile angezeigt.
\end{frame}

\begin{frame}[t]{Beispielinhalt}
    Bei Frames mit Option \texttt{[t]} werden die Inhalte nicht vertikal zentriert, sondern an der Oberkante begonnen.
\end{frame}


\begin{frame}{Literatur}
  \begin{exampleblock}{Backup-Teil}
      Folien, die nach \texttt{\textbackslash beginbackup} eingefügt werden, zählen nicht in die Gesamtzahl der Folien.
  \end{exampleblock}

  \printbibliography
\end{frame}

\section{Farben}
%% ----------------------------------------
%% | Test-Folie mit definierten Farben |
%% ----------------------------------------
\begin{frame}{Farbpalette}
\tiny

% GREEN
  \colorbox{kit-green100}{kit-green100}
  \colorbox{kit-green90}{kit-green90}
  \colorbox{kit-green80}{kit-green80}
  \colorbox{kit-green70}{kit-green70}
  \colorbox{kit-green60}{kit-green60}
  \colorbox{kit-green50}{kit-green50}
  \colorbox{kit-green40}{kit-green40}
  \colorbox{kit-green30}{kit-green30}
  \colorbox{kit-green25}{kit-green25}
  \colorbox{kit-green20}{kit-green20}
  \colorbox{kit-green15}{kit-green15}
  \colorbox{kit-green10}{kit-green10}
  \colorbox{kit-green5}{kit-green5}

% BLUE
  \colorbox{kit-blue100}{kit-blue100}
  \colorbox{kit-blue90}{kit-blue90}
  \colorbox{kit-blue80}{kit-blue80}
  \colorbox{kit-blue70}{kit-blue70}
  \colorbox{kit-blue60}{kit-blue60}
  \colorbox{kit-blue50}{kit-blue50}
  \colorbox{kit-blue40}{kit-blue40}
  \colorbox{kit-blue30}{kit-blue30}
  \colorbox{kit-blue25}{kit-blue25}
  \colorbox{kit-blue20}{kit-blue20}
  \colorbox{kit-blue15}{kit-blue15}
  \colorbox{kit-blue10}{kit-blue10}
  \colorbox{kit-blue5}{kit-blue5}

% RED
  \colorbox{kit-red100}{kit-red100}
  \colorbox{kit-red90}{kit-red90}
  \colorbox{kit-red80}{kit-red80}
  \colorbox{kit-red70}{kit-red70}
  \colorbox{kit-red60}{kit-red60}
  \colorbox{kit-red50}{kit-red50}
  \colorbox{kit-red40}{kit-red40}
  \colorbox{kit-red30}{kit-red30}
  \colorbox{kit-red25}{kit-red25}
  \colorbox{kit-red20}{kit-red20}
  \colorbox{kit-red15}{kit-red15}
  \colorbox{kit-red10}{kit-red10}
  \colorbox{kit-red5}{kit-red5}

% GREY
  \colorbox{kit-gray100}{\color{white}kit-gray100}
  \colorbox{kit-gray90}{\color{white}kit-gray90}
  \colorbox{kit-gray80}{\color{white}kit-gray80}
  \colorbox{kit-gray70}{\color{white}kit-gray70}
  \colorbox{kit-gray60}{\color{white}kit-gray60}
  \colorbox{kit-gray50}{\color{white}kit-gray50}
  \colorbox{kit-gray40}{kit-gray40}
  \colorbox{kit-gray30}{kit-gray30}
  \colorbox{kit-gray25}{kit-gray25}
  \colorbox{kit-gray20}{kit-gray20}
  \colorbox{kit-gray15}{kit-gray15}
  \colorbox{kit-gray10}{kit-gray10}
  \colorbox{kit-gray5}{kit-gray5}

% Orange
  \colorbox{kit-orange100}{kit-orange100}
  \colorbox{kit-orange90}{kit-orange90}
  \colorbox{kit-orange80}{kit-orange80}
  \colorbox{kit-orange70}{kit-orange70}
  \colorbox{kit-orange60}{kit-orange60}
  \colorbox{kit-orange50}{kit-orange50}
  \colorbox{kit-orange40}{kit-orange40}
  \colorbox{kit-orange30}{kit-orange30}
  \colorbox{kit-orange25}{kit-orange25}
  \colorbox{kit-orange20}{kit-orange20}
  \colorbox{kit-orange15}{kit-orange15}
  \colorbox{kit-orange10}{kit-orange10}
  \colorbox{kit-orange5}{kit-orange5}

% lightgreen
  \colorbox{kit-lightgreen100}{kit-lightgreen100}
  \colorbox{kit-lightgreen90}{kit-lightgreen90}
  \colorbox{kit-lightgreen80}{kit-lightgreen80}
  \colorbox{kit-lightgreen70}{kit-lightgreen70}
  \colorbox{kit-lightgreen60}{kit-lightgreen60}
  \colorbox{kit-lightgreen50}{kit-lightgreen50}
  \colorbox{kit-lightgreen40}{kit-lightgreen40}
  \colorbox{kit-lightgreen30}{kit-lightgreen30}
  \colorbox{kit-lightgreen25}{kit-lightgreen25}
  \colorbox{kit-lightgreen20}{kit-lightgreen20}
  \colorbox{kit-lightgreen15}{kit-lightgreen15}
  \colorbox{kit-lightgreen10}{kit-lightgreen10}
  \colorbox{kit-lightgreen5}{kit-lightgreen5}

% Brown
  \colorbox{kit-brown100}{kit-brown100}
  \colorbox{kit-brown90}{kit-brown90}
  \colorbox{kit-brown80}{kit-brown80}
  \colorbox{kit-brown70}{kit-brown70}
  \colorbox{kit-brown60}{kit-brown60}
  \colorbox{kit-brown50}{kit-brown50}
  \colorbox{kit-brown40}{kit-brown40}
  \colorbox{kit-brown30}{kit-brown30}
  \colorbox{kit-brown25}{kit-brown25}
  \colorbox{kit-brown20}{kit-brown20}
  \colorbox{kit-brown15}{kit-brown15}
  \colorbox{kit-brown10}{kit-brown10}
  \colorbox{kit-brown5}{kit-brown5}

% Purple
  \colorbox{kit-purple100}{kit-purple100}
  \colorbox{kit-purple90}{kit-purple90}
  \colorbox{kit-purple80}{kit-purple80}
  \colorbox{kit-purple70}{kit-purple70}
  \colorbox{kit-purple60}{kit-purple60}
  \colorbox{kit-purple50}{kit-purple50}
  \colorbox{kit-purple40}{kit-purple40}
  \colorbox{kit-purple30}{kit-purple30}
  \colorbox{kit-purple25}{kit-purple25}
  \colorbox{kit-purple20}{kit-purple20}
  \colorbox{kit-purple15}{kit-purple15}
  \colorbox{kit-purple10}{kit-purple10}
  \colorbox{kit-purple5}{kit-purple5}

% Cyan
  \colorbox{kit-cyan100}{kit-cyan100}
  \colorbox{kit-cyan90}{kit-cyan90}
  \colorbox{kit-cyan80}{kit-cyan80}
  \colorbox{kit-cyan70}{kit-cyan70}
  \colorbox{kit-cyan60}{kit-cyan60}
  \colorbox{kit-cyan50}{kit-cyan50}
  \colorbox{kit-cyan40}{kit-cyan40}
  \colorbox{kit-cyan30}{kit-cyan30}
  \colorbox{kit-cyan25}{kit-cyan25}
  \colorbox{kit-cyan20}{kit-cyan20}
  \colorbox{kit-cyan15}{kit-cyan15}
  \colorbox{kit-cyan10}{kit-cyan10}
  \colorbox{kit-cyan5}{kit-cyan5}
    
\end{frame}
%% ----------------------------------------
%% | /Test-Folie mit definierten Farben |
%% ----------------------------------------
\backupend

\end{document}