%% Beispiel-Präsentation mit LaTeX Beamer im KIT-Design
%% entsprechend den Gestaltungsrichtlinien vom 1. August 2020
%%
%% Siehe https://sdqweb.ipd.kit.edu/wiki/Dokumentvorlagen

%% Beispiel-Präsentation
\documentclass{sdqbeamer} 
 
%% Titelbild
\titleimage{banner_2020_kit}

%% Gruppenlogo
%\grouplogo{mylogo} 

%% Gruppenname und Breite (Standard: 50 mm)
\groupname{Department of Informatics -- Institute of Information Security and Dependability (KASTEL)}
%\groupnamewidth{50mm}

% Beginn der Präsentation

\title[Rust \& Refinement Types]{Corten: Refinement Types for Imperative Languages with Ownership}
\subtitle{Abschlusspräsentation Masterarbeit} 
\author[Carsten Csiky]{Carsten Csiky}

\date[26.\,10.\,2022]{26th Oktober 2022}

% Literatur 
 
\usepackage[citestyle=authoryear,bibstyle=numeric,hyperref,backend=biber]{biblatex}
\addbibresource{presentation.bib}
\bibhang1em

\usepackage{minted}
\usepackage{mathtools}
\usemintedstyle{colorful}

\usepackage{mathpartir}
\usepackage[overridenote, notesposition=right]{pdfpc}
\newcommand{\code}[1]{\texttt{#1}}

\begin{document}
 
%Titelseite
\KITtitleframe

%Inhaltsverzeichnis
\begin{frame}{Inhaltsverzeichnis}
\tableofcontents
\end{frame}

\section{Motivation}

\begin{frame}[fragile]{Motivation}{}
 \begin{minted}[autogobble]{rust}
		fn max(a: i32, b: i32) {
			if a > b { a } else { b }
		}
 \end{minted}
 \onslide<2->{
	 $$
		\text{Return Value }(v): v \geq a \wedge v \geq b
	 $$
 }
 \onslide<3->{
	Refinement Types\cite{rondon_liquid_2008} in Functional Programming Languages
 }
 \note{
	- In BA: FP-Patterns in imperativen Programmier Sprachen.\\
	- Aufgefallen: \\
		- Eigentliches Ziel: nicht so schwer, aber\\
		- Abwesenheit von unabhängigen Änderungen verbraucht großteil der Zeit\\
	- Also: welche alternativen gibt es?\\
	- Refinement Types
 }
\end{frame}


\begin{frame}[fragile]{Motivation}{}
	\begin{minted}[autogobble]{rust}
		//@ max(a: i32, b: i32) -> {v:i32 | v >= a && v >= b }
		fn  max(a: i32, b: i32) -> i32 {
			if a > b { a } else { b }
		}
	\end{minted}
	\note{
		Wie funktioniert das?\\
		\\
		1. Extra Typ Spezifikation \\
		2. Ausdruck in klammern ähnlich zu set compresension (erlaubte werte) \\
		\\
		=> Wie werden diese typen überprüft?
	}
\end{frame}


\begin{frame}[fragile]{Motivation}
	\begin{minted}[autogobble]{rust}
		//@ max(a: i32, b: i32) -> {v:i32 | v >= a && v >= b }
		fn  max(a: i32, b: i32) -> i32 {
			if a > b { a } else { b }
		}
	\end{minted}
	\begin{align*}
		&\text{let } \Gamma = (a : \{ v : \code{i32} \mid \text{true}\}, b : \{ v : \code{i32} \mid \text{true}\}) \text{ and } \tau = 
		\{ v: \code{i32} \mid v \geq a \wedge v \geq b \}
		\\
		&\inferrule*
			{
				\onslide<2->{
					\inferrule*
					{ \onslide<4->{
							\inferrule*
							{\star}
							{\Gamma, a > b \vdash a : \{ v : \code{i32} \mid v \doteq a\}}
						}
						\\
							\onslide<3->{
								\inferrule*
								{ 
									\onslide<5->{
										\text{SMT-VALID}
										\footnotesize{
											\left(
											\begin{aligned}[]
												& \, \text{true} \wedge \text{true} \wedge a > b \\
												&\ \wedge v \doteq a \\
												&\quad \implies (v \geq a \wedge v \geq b)
											\end{aligned}
											\right)
										}
									}
								}
								{\Gamma, a > b \vdash \{ v : \code{i32} \mid v \doteq a\} \preceq \tau}
							}
					}
					{
						\Gamma, a > b \vdash a : \tau
					}
				}
				\\
					\onslide<2->{
						\inferrule*
						{\onslide<6->{\vdots}}
						{\Gamma, \neg(a > b) \vdash b : \tau}
					}
			}
			{\Gamma \vdash \code{if } a > b\ \{ a \} \code{ else } \{ b \} : \tau}
	\end{align*}
	\note{
		0. Ziel: Prüfe Body gegen signatur \\
		1. Abkürzungen \\
		2. Funktionskörper muss return type haben; Ctx enthält die argumente (uneingeschränkt -> true predicate) \\
		3. Beide Seiten: if und else \\
			- Füge path condition zu context hinzu \\
		4. Zentrale Idee: Subtyp einführen, der bekannt ist \\
		5. Spezifischerer Type folg trivialerweise aus Regeln \\
		6. Was bedeutet Subtyp? Predikate müssen sich implizieren (in CTX) \\
		\\
		---- \\
		\\
		Anmerken: Zusammenarbeit von Type-System (für grobes) und SMT / Logic für kompliziertes
		\\
		Funktioniert gut, aber was ist mit mutabillity?
	}
\end{frame}



\begin{frame}[fragile]{Motivation}{}
	\begin{columns}
		\column{.5\textwidth}
		\begin{minted}[autogobble]{rust}
			fn  clamp(a: &mut i32, b: i32) {
				if *a > b { *a = b }
			}
		\end{minted}
		\begin{onlyenv}<2->
			\begin{minted}[autogobble]{rust}
				fn  client(...) {
					...
					clamp(&mut x, 5);
					clamp(&mut y, 6);
					print(x);
					...
				}
			\end{minted}
		\end{onlyenv}

		\column{.5\textwidth}<3->
		What does this it \code{print(x)} output?
		\begin{itemize}
			\item Could be: old \code{x} or \code{5}
			\item<4-> But also \code{6} (if x aliases with y)!
		\end{itemize}
	\end{columns}
	\note{
		- Hier: `clamp` stellt sicher, dass der referenzierte wert max `b` groß ist \\
		- Neu: `client`. Zeigt zusätzliche schwierigkeiten.\\
		-  Zwei Aufrufe an clamp. \\
			- Welche Werte könnte `x, y` haben? \\
			- Auf jeden Fall mal: `x` und `5` (je nachdem was größer) \\
			- Aber! auch 6
	}
\end{frame}

\section{Empirical Analysis}
\section{Solution}
\section{Soundness Justification}
\section{Related Work}
\section{Conclusion / Future Work}


\appendix
\beginbackup

\begin{frame}{Literatur}
\begin{exampleblock}{Backup-Teil}
    Folien, die nach \texttt{\textbackslash beginbackup} eingefügt werden, zählen nicht in die Gesamtzahl der Folien.
\end{exampleblock}

\printbibliography
\end{frame}


\subsection{Erster Unterabschnitt}
\begin{frame}{Blöcke}{in den KIT-Farben}
	\begin{columns}
		\column{.3\textwidth}
		\begin{greenblock}{Greenblock}
			Standard (\texttt{block})
        \end{greenblock}
		\column{.3\textwidth}
		\begin{blueblock}{Blueblock}
			= \texttt{exampleblock}
        \end{blueblock}
		\column{.3\textwidth}
		\begin{redblock}{Redblock}
			= \texttt{alertblock}
        \end{redblock}
	\end{columns}
	\begin{columns}
		\column{.3\textwidth}
        \begin{brownblock}{Brownblock}
        \end{brownblock}
		\column{.3\textwidth}
        \begin{purpleblock}{Purpleblock}
        \end{purpleblock}
		\column{.3\textwidth}
        \begin{cyanblock}{Cyanblock}
        \end{cyanblock}
	\end{columns}
	\begin{columns}
		\column{.3\textwidth}
        \begin{yellowblock}{Yellowblock}
        \end{yellowblock}
		\column{.3\textwidth}
        \begin{lightgreenblock}{Lightgreenblock}
        \end{lightgreenblock}
		\column{.3\textwidth}
        \begin{orangeblock}{Orangeblock}
        \end{orangeblock}
	\end{columns}
	\begin{columns}
		\column{.3\textwidth}
        \begin{grayblock}{Grayblock}
        \end{grayblock}
		\column{.3\textwidth}
		\begin{contentblock}{Contentblock}
			(farblos)
		\end{contentblock}
		\column{.3\textwidth}
	\end{columns}
\end{frame}
	  
\subsection{Zweiter Unterabschnitt}
\begin{frame}{Auflistungen}
	Text
	\begin{itemize}
		\item Auflistung\\ Umbruch
		\item Auflistung
		\begin{itemize}
			\item Auflistung
			\item Auflistung
		\end{itemize}
	\end{itemize}
\end{frame}

\section{Zweiter Abschnitt}

\begin{frame}
        Bei Frames ohne Titel wird die Kopfzeile nicht angezeigt, und  
    der freie Platz kann für Inhalte genutzt werden.
\end{frame}

\begin{frame}[plain]
    Bei Frames mit Option \texttt{[plain]} werden weder Kopf- noch Fußzeile angezeigt.
\end{frame}

\begin{frame}[t]{Beispielinhalt}
    Bei Frames mit Option \texttt{[t]} werden die Inhalte nicht vertikal zentriert, sondern an der Oberkante begonnen.
\end{frame}


\begin{frame}{Beispielinhalt: Literatur}
\end{frame}

\section{Farben}
%% ----------------------------------------
%% | Test-Folie mit definierten Farben |
%% ----------------------------------------
\begin{frame}{Farbpalette}
\tiny

% GREEN
	\colorbox{kit-green100}{kit-green100}
	\colorbox{kit-green90}{kit-green90}
	\colorbox{kit-green80}{kit-green80}
	\colorbox{kit-green70}{kit-green70}
	\colorbox{kit-green60}{kit-green60}
	\colorbox{kit-green50}{kit-green50}
	\colorbox{kit-green40}{kit-green40}
	\colorbox{kit-green30}{kit-green30}
	\colorbox{kit-green25}{kit-green25}
	\colorbox{kit-green20}{kit-green20}
	\colorbox{kit-green15}{kit-green15}
	\colorbox{kit-green10}{kit-green10}
	\colorbox{kit-green5}{kit-green5}

% BLUE
	\colorbox{kit-blue100}{kit-blue100}
	\colorbox{kit-blue90}{kit-blue90}
	\colorbox{kit-blue80}{kit-blue80}
	\colorbox{kit-blue70}{kit-blue70}
	\colorbox{kit-blue60}{kit-blue60}
	\colorbox{kit-blue50}{kit-blue50}
	\colorbox{kit-blue40}{kit-blue40}
	\colorbox{kit-blue30}{kit-blue30}
	\colorbox{kit-blue25}{kit-blue25}
	\colorbox{kit-blue20}{kit-blue20}
	\colorbox{kit-blue15}{kit-blue15}
	\colorbox{kit-blue10}{kit-blue10}
	\colorbox{kit-blue5}{kit-blue5}

% RED
	\colorbox{kit-red100}{kit-red100}
	\colorbox{kit-red90}{kit-red90}
	\colorbox{kit-red80}{kit-red80}
	\colorbox{kit-red70}{kit-red70}
	\colorbox{kit-red60}{kit-red60}
	\colorbox{kit-red50}{kit-red50}
	\colorbox{kit-red40}{kit-red40}
	\colorbox{kit-red30}{kit-red30}
	\colorbox{kit-red25}{kit-red25}
	\colorbox{kit-red20}{kit-red20}
	\colorbox{kit-red15}{kit-red15}
	\colorbox{kit-red10}{kit-red10}
	\colorbox{kit-red5}{kit-red5}

% GREY
	\colorbox{kit-gray100}{\color{white}kit-gray100}
	\colorbox{kit-gray90}{\color{white}kit-gray90}
	\colorbox{kit-gray80}{\color{white}kit-gray80}
	\colorbox{kit-gray70}{\color{white}kit-gray70}
	\colorbox{kit-gray60}{\color{white}kit-gray60}
	\colorbox{kit-gray50}{\color{white}kit-gray50}
	\colorbox{kit-gray40}{kit-gray40}
	\colorbox{kit-gray30}{kit-gray30}
	\colorbox{kit-gray25}{kit-gray25}
	\colorbox{kit-gray20}{kit-gray20}
	\colorbox{kit-gray15}{kit-gray15}
	\colorbox{kit-gray10}{kit-gray10}
	\colorbox{kit-gray5}{kit-gray5}

% Orange
	\colorbox{kit-orange100}{kit-orange100}
	\colorbox{kit-orange90}{kit-orange90}
	\colorbox{kit-orange80}{kit-orange80}
	\colorbox{kit-orange70}{kit-orange70}
	\colorbox{kit-orange60}{kit-orange60}
	\colorbox{kit-orange50}{kit-orange50}
	\colorbox{kit-orange40}{kit-orange40}
	\colorbox{kit-orange30}{kit-orange30}
	\colorbox{kit-orange25}{kit-orange25}
	\colorbox{kit-orange20}{kit-orange20}
	\colorbox{kit-orange15}{kit-orange15}
	\colorbox{kit-orange10}{kit-orange10}
	\colorbox{kit-orange5}{kit-orange5}

% lightgreen
	\colorbox{kit-lightgreen100}{kit-lightgreen100}
	\colorbox{kit-lightgreen90}{kit-lightgreen90}
	\colorbox{kit-lightgreen80}{kit-lightgreen80}
	\colorbox{kit-lightgreen70}{kit-lightgreen70}
	\colorbox{kit-lightgreen60}{kit-lightgreen60}
	\colorbox{kit-lightgreen50}{kit-lightgreen50}
	\colorbox{kit-lightgreen40}{kit-lightgreen40}
	\colorbox{kit-lightgreen30}{kit-lightgreen30}
	\colorbox{kit-lightgreen25}{kit-lightgreen25}
	\colorbox{kit-lightgreen20}{kit-lightgreen20}
	\colorbox{kit-lightgreen15}{kit-lightgreen15}
	\colorbox{kit-lightgreen10}{kit-lightgreen10}
	\colorbox{kit-lightgreen5}{kit-lightgreen5}

% Brown
	\colorbox{kit-brown100}{kit-brown100}
	\colorbox{kit-brown90}{kit-brown90}
	\colorbox{kit-brown80}{kit-brown80}
	\colorbox{kit-brown70}{kit-brown70}
	\colorbox{kit-brown60}{kit-brown60}
	\colorbox{kit-brown50}{kit-brown50}
	\colorbox{kit-brown40}{kit-brown40}
	\colorbox{kit-brown30}{kit-brown30}
	\colorbox{kit-brown25}{kit-brown25}
	\colorbox{kit-brown20}{kit-brown20}
	\colorbox{kit-brown15}{kit-brown15}
	\colorbox{kit-brown10}{kit-brown10}
	\colorbox{kit-brown5}{kit-brown5}

% Purple
	\colorbox{kit-purple100}{kit-purple100}
	\colorbox{kit-purple90}{kit-purple90}
	\colorbox{kit-purple80}{kit-purple80}
	\colorbox{kit-purple70}{kit-purple70}
	\colorbox{kit-purple60}{kit-purple60}
	\colorbox{kit-purple50}{kit-purple50}
	\colorbox{kit-purple40}{kit-purple40}
	\colorbox{kit-purple30}{kit-purple30}
	\colorbox{kit-purple25}{kit-purple25}
	\colorbox{kit-purple20}{kit-purple20}
	\colorbox{kit-purple15}{kit-purple15}
	\colorbox{kit-purple10}{kit-purple10}
	\colorbox{kit-purple5}{kit-purple5}

% Cyan
	\colorbox{kit-cyan100}{kit-cyan100}
	\colorbox{kit-cyan90}{kit-cyan90}
	\colorbox{kit-cyan80}{kit-cyan80}
	\colorbox{kit-cyan70}{kit-cyan70}
	\colorbox{kit-cyan60}{kit-cyan60}
	\colorbox{kit-cyan50}{kit-cyan50}
	\colorbox{kit-cyan40}{kit-cyan40}
	\colorbox{kit-cyan30}{kit-cyan30}
	\colorbox{kit-cyan25}{kit-cyan25}
	\colorbox{kit-cyan20}{kit-cyan20}
	\colorbox{kit-cyan15}{kit-cyan15}
	\colorbox{kit-cyan10}{kit-cyan10}
	\colorbox{kit-cyan5}{kit-cyan5}
		
\end{frame}
%% ----------------------------------------
%% | /Test-Folie mit definierten Farben |
%% ----------------------------------------
\backupend

\end{document}